\chapter{Analiza problemu}
Dostep do internetu - posiadanie na telefonie bazy danych wszystkich lokalizacji sieci wifi jest dosyć nierealistyczne ponieważ rozmiar danych dla warszawy to byłoby około 40MB danych przedstawionych w formacie CSV lub 9MB dla spakowanego archiwum tych danych. Dodaktowo dochodzi element obliczania lokalizacji na tak rozległych danych. Gdy natomiast, wziąć pod uwagę bazę stacji GSM. To taka baza jest dużo mniejsza, oraz jej dane nie zmieniają się tak często. Taką bazę można umieszczać na urządzeniach by ustalały lokalizację w sposób stuprocentowy offline i anonimowo. Niestety lokalizacja na podstawie puntków stacji GSM jest niedokładna. Średnia dokładność dla lokalizacji ustalonej na bazie nadajników GSM to 3000m. Gdzie dla sieci wifi to 30m

%TODO? jakieś źródła tych dokładności lokalizacji by się przydały. Np. na podstawie 

Podczas gdy użytkownik ma dostęp do internetu, to można go zlokalizować na podstawie adresu ip. Podanych sieci wifi które ma w zasięgu. Podanych stacji GSM które ma w zasięgu.

%TODO: Skompresowane dane do lokalizowania się w Warszawie? Czy możliwe jest zapisanie lokalizacji sieci wifi w taki sposób aby zajmowały mało miejsca, nie były kosztowne do obliczenia, offline?
\section{Przetwarzanie danych osobowych}

\subsection{Standard \_nomap}
Google zaproponowało dopisywanie do końca nazwy sieci _nomap aby ich urządzenia nie zbierały danych
\subsection{Zbieranie nazw sieci wifi}
Prawo w stanach zabrania
%jakieś źródła czy cuś

\subsubsection{Problem z określeniem lokalizacji}
Ze względu na lokalizację między dwoma punktami klient-serwer. Nie jest możliwym stwierdzenie z jakiego kierunku przyszedł sygnał. Co przy mapowaniu przy użyciu telefonu komórkowego spodowuje umiejscowienie wszystkich sieci wifi na pozycji telefonu. Np. na środku ulicy lub na chodniku.
% widać to świetnie na tej mapie: http://owm.vreeken.net/map/
Rozwiązaniem droższym byłoby użycie kilku urządzeń, najlepiej z antenami kierunkowymi, dzięki czemu moglibyśmy stwierdzić czy stacja bazowa znajduje się po lewej stronie ulicy czy po prawej na podstawie siły sygnału.
Dzięki technologii MIMO można określić z którego kierunku przyszedł sygnał na jednym urządzeniu.

\section{Lokalizacja sieci wifi a wpa.darkircop.org}
Na stronie wpa.darkircop.org możemy znaleźć 

\section{Geografia}
Na potrzeby późniejszych rozdziałów, przyjąłem jako punkt centralny Warszawy z Wikipedii jako współrzędne geograficzne w formacie DM S (Stopnie:Minuty:Sekundy) - 52° 13′ 56″ N, 21° 0′ 30″ E - lub odpowiadające im w formacie DM F (Stopnie Setne zwane miaro miernymi) 52.232222, 21.008333

