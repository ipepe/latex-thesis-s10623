\chapter{Zagadnienia wykorzystania szablonu}

Ponieważ pisana praca inżynierska jest pracą techniczną, trzeba zachować pewne standardy techniczne charakterystyczne dla prac tego typu. Jednym z przykładów jest odwoływanie się do wszystkich:
\begin{itemize}
    \item obrazków,
    \item rysunków,
    \item schamatów,
    \item tabel,
    \item wykresów,
    \item \textit{itp}
\end{itemize}
Operację tę realizujemy z wykorzystaniem polecenia \texttt{\textbackslash ref}, i tak np gdy chcemy odwołać się do poniższego rysunku mówimy że: \textit{to i to zagadnienie jest ilustrowane przez rysunek \ref{fig:mobile}}. Możliwe są równie odwołanie się do konkretnego rozdziału, pod rozdziału czy punktu, np: \textit{ przykład wypunktowania alfabetycznego pokazano w podrozdziale \ref{ss:ex1}}.

Kolejnym bardzo ważnym elementem pracy technicznej jest wskazanie wszystkich źródeł jakie zostały wykorzystane w Państwa pracy. W tym przypadku wykorzystywany jest plik o rozszerzenie \texttt{.bib} oraz polecenie \texttt{\textbackslash cite} przykładowo odwołuję się do publikacji.\cite{Tomaszewski2000}


\begin{figure}[h!]
  \centering
    \includegraphics[width=0.5\textwidth]{images/mobile-banner}
  \caption{Close-up of a gull}
  \label{fig:mobile}
\end{figure}

\lipsum[1]

\section{Structure}
\lipsum[2]

\subsection{Top Matter}
\lipsum[1]

\subsubsection{Wypunktowanie}
\label{ss:ex1}

Przykład wypunktowania:

\begin{description}
	\item[(a)] minimalizacja liczby niezależnych parametrów funkcji plenoptycznej;
	\item[(b)] identyfikacja schematu próbkowania plenoptycznego w modelu kamery otworkowej;
	\item[(c)] identyfikacja ruchu wykonywanego przez element plenoptyczny;
	\item[(d)] filtracja funkcji plenoptycznej w systemach percepcji wizualnej.
\end{description}
\lipsum[1]

\section{Minimalizacja parametrów funkcji plenoptycznej}
\label{sec:mpfp}