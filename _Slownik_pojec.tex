\chapter*{Słownik pojęć}

\newcommand{\entry}[2]{\noindent\textbf{#1}\ $\bullet$\ {#2}}

\entry{ORM}{skrótowe oznaczenie dla "mapowanie obiektowo-relacyjne" (od angielskiego Object-Relational Mapping).}

\entry{Punkt dostępu (od ang. Accesspoint)}{Stacja bazowa sieci bezprzewodowej}

\entry{Aplikacja internetowa}{Aplikacja WWW/HTTP (webowa)}

\entry{SSID}{Nazwa punktu dostępu}

\entry{BSSID}{adres MAC stacji bazowej sieci bezprzewodowej}
%\entry{Interfejs}{Międzymordzie}

\entry{Klastrowanie}{Zadaniem klastrowania jest pogrupowanie zbioru obserwacji w klastry tak, aby w ramach klastra obserwacje były do siebie, jak najbardziej podobne, a jednocześnie jak najbardziej różne od obserwacji w innych klastrach.}
%https://eduwiki.wmi.amu.edu.pl/klastrowanie

\entry{Kanał (w kontekście sieci bezprzewodowych i punktów dostępów)}{Sieci bezprzewodowe operują na określonych częstotliwościach. W przypadku specyfikacji IEEE 802.11g jest ich 14. Te poszczególne częstotliwości nazywamy kanałami.}

\entry{wardriving}{Jest to czynność, która polega na zbieraniu jak największej ilości sieci bezprzewodowych. Wielokrotnie opisywana krytycznie ze względu na pomyłkę w intencji osoby wykonującej zbieranie informacji. Często mylona z uzyskiwaniem dostępu do sieci bezprzewodowych. \cite{WardrivingWiki}}

\entry{MIMO (od ang. Multiple-Input-Multiple-Output)}{rozwiązanie, w którym karta sieciowa posiada wiele wyjść i wiele wejść, przez co można nadawać i odbierać na specyficznym Wejściu/Wyjściu, które posiada lepsze parametry komunikacji}

\entry{denormalizacja}{jest to wprowadzenie kontrolowanej nadmierności do bazy danych w celu przyśpieszenia wykonywania na niej operacji (np. obsługiwania zapytań); dzięki denormalizacji bazy unika się kosztownych operacji połączeń tabel\cite{DenormalizacjaWiki}}
