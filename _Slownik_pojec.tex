\chapter*{Słownik pojęć}

\newcommand{\entry}[2]{\textbf{#1}\ $\bullet$\ {#2}}

\entry{ORM}{skrótowe oznaczenie dla "mapowanie obiektowo-relacyjne" (od angielskiego Object-Relational Mapping).}

\entry{Punkt dostępu (od ang. Accesspoint)}{Stacja bazowa WIFI.}
\entry{Aplikacja internetowa}{Aplikacja WWW/HTTP (webowa)}
\entry{SSID}{Nazwa punktu dostępu}
\entry{BSSID}{Mac Address}
\entry{Interfejs}{Międzymordzie}
\entry{Klastrowanie}{Zadaniem klastrowania jest pogrupowanie zbioru obserwacji w klastry tak, aby w ramach klastra obserwacje były do siebie jak najbardziej podobne, a jednocześnie jak najbardziej różne od obserwacji w innych klastrach.}
%https://eduwiki.wmi.amu.edu.pl/klastrowanie

\entry{Kanał (w kontekście sieci przezprzewodowych i punktów dostępów)}{WIFI operuje na określonych częstotliwościach. W przypadku 802.11g jest ich 14. Te poszczególne częstotliwości nazywamy kanałami.}

\entry{wardriving}{"zbieranie" jak najwiekszej ilosci sieci wifi.}
%https://en.wikipedia.org/wiki/Wardriving

\entry{MIMO (od ang. Multiple-Input-Multiple-Output)}{rozwiązanie w którym karta sieciowa posiada wiele wyjść i wiele wejść, przez co można nadawać i odbierać na specyficznym Wejściu/Wyjściu które posiada lepsze parametry komunikacji}

\entry{denormalizacja}{jest to wprowadzenie kontrolowanej nadmierności do bazy danych w celu przyśpieszenia wykonywania na niej operacji (np. obsługiwania zapytań); dzięki denormalizacji bazy unika się kosztownych operacji połączeń tabel}
%https://pl.wikipedia.org/wiki/Denormalizacja_bazy_danych

