\chapter{Część praktyczna}
W rozdziale tym przedstawię proces utworzenia aplikacji internetowej oraz mobilnej. Omówię wyzwania oraz problemy, z którymi się spotkałem podczas implementacji. Opiszę też zasadę działania algorytmów potrzebnych do realizacji podstawowych założeń aplikacji mobilnej oraz internetowej. Na koniec przeprowadzę badania i wyciągnę wnioski, które skonfrontuje z postawioną tezą.

\section{Przygotowanie}
Aby rozpocząć prace nad utworzeniem aplikacji internetowej i mobilnej potrzebny jest warsztat sprzętowy oraz odpowiedniego oprogramowania zainstalowanego i skonfigurowanego na danym sprzęcie.

\subsection{Sprzęt}
Do wytworzenia aplikacji internetowej oraz mobilnej potrzebny jest komputer. W moim przypadku środowisko programistyczne zainstalowałem i skonfigurowałem na MacBooku Pro z 2011 roku. Do celów badawczych pracy potrzebny mi był smartfon z systemem operacyjnym Android. Wykorzystałem do tego zadania LG Nexus 5. Dodatkowo, jeżeli mamy smartfon z systemem Android to nie musimy konfigurować emulatora, co w przypadku mojego komputera mogłoby bardzo ograniczyć zasoby pamięci RAM.

\subsection{Oprogramowanie}
\subsubsection{Git}
Korzystając z nowoczesnych narzędzi wytwarzania oprogramowania, naturalne jest korzystanie z systemu kontroli wersji. Osobiście mam doświadczenie z systemem Git. Aby korzystać z takiego systemu, potrzebujemy mieć zainstalowane narzędzie klienckie. Ponadto potrzebujemy mieć skonfigurowane repozytoria na odpowiednim serwerze. Ja do tego celu wykorzystałem darmowy plan na platformie Github. Utworzyłem dwa repozytoria, oddzielne dla aplikacji internetowej i aplikacji mobilnej.

\subsubsection{Ruby}
Spośród wielu implementacji oraz narzędzi do zarządzania środowiskiem uruchomieniowym Ruby wybrałem wersję 2.3.3, która na dzień tworzenia projektu była najnowsza. Do zarządzania wersją Ruby wykorzystuję narzędzie rbenv.

\subsubsection{RubyMine}
Mimo że Ruby jest językiem interpretowanym, a zestaw konsolowych narzędzi Ruby on Rails wykorzystuje się w sposób naturalny. Jednak osobiście bardzo lubię korzystać z tego edytora. Podświetlanie składni, refaktoryzacja kodu i integracja z Gitem powodują, że pisanie aplikacji internetowej nie jest toporne i żmudne. Wersja na dzień utworzenia projektu to 2017.1.

\subsubsection{PostgreSQL}
PostgreSQL na środowiska Mac OS można ściągnąć w postaci zwykłej aplikacji systemowej i tak też zrobiłem w swoim projekcie. Aplikacja zasugerowała najnowszą wersję systemu bazy danych i na dzień utworzenia projektu oznaczenie wersji to 9.6.

\subsubsection{Android Studio}
Najprostszym sposobem wytwarzania oprogramowania dla systemów Android jest korzystanie z pakietu Android Studio. Najnowsza wersja na dzień utworzenia projektu to 2.3.2. W ramach instalacji tego środowiska programistycznego zostało zainstalowanych wiele bibliotek. Między innymi SDK Manager, w którym zainstalowałem SDK Platformy o oznaczeniu Android 7.1.1 (Nougat), której poziom API to 25.

\subsubsection{Przeglądarka internetowa}
Gdy mowa o aplikacji internetowej, to nie można zapomnieć o jej kliencie — przeglądarce internetowej. Przy rozwoju aplikacji opartej na Ruby on Rails, nie ma większego znaczenia, z jakiej przeglądarki skorzystamy. Osobiście jednak polecam, aby na komputerze posiadać przeglądarki kilku firm, ze względu na kompatybilność. Jak się dowiemy w późniejszych rozdziałach, ma to znaczenie przy rozwoju takiej aplikacji internetowej.

\section{Aplikacja internetowa}
Implementację rozpocząłem od importu istniejących danych o lokalizacjach sieci WIFI z różnych serwisów. Potem skupiłem się na prezentacji zaimportowanych danych. Na koniec stworzyłem ścieżkę, pod którą można geolokalizować na podstawie podanych adresów MAC i siły sygnałów podanych sieci bezprzewodowych.

\subsection{Struktura danych}
W internecie można znaleźć trzy serwisy, które za darmo udostępniają zrzuty bazy z lokalizacjami bezprzewodowych sieci.

\begin{table}
\caption{Rozmiary i format zrzutów baz danych z informacjami o lokalizacjach sieci bezprzewodowych}
\label{table:wifilocationdatabases}
\begin{tabular}{ |l|l|l|  }
\hline
Adres serwisu & Rozmiar danych & Fromat \\
\hline
\hline
openwifi.su (openwlanmap.org) & 245,7MB & CSV spakowany do tar.bz2 \\
\hline
mylnikov.org & 657 MB & CSV spakowany do zip \\
\hline
radiocells.org (openbmap) & 1.6GB & baza danych sqlite \\
\hline
\end{tabular}
\end{table}

Na podstawie importowanych danych przygotowałem strukturę bazy danych obserwacji. Uwzględniłem też kolumny, które będą potrzebne przy zbieraniu obserwacji z aplikacji mobilnej. Tablica \ref{table:dbscheme}

\begin{table}
\caption{Kolumny i typy przechowywania danych wraz z ich znaczeniem w aplikacji internetowej}
\label{table:dbscheme}
\begin{tabular} { |l|l|l|p{6cm}|  }
\hline
Typ & Nazwa & Dodatkowe uwagi & Znaczenie (opcjonalne) \\
\hline
\hline
string   & bssid & null: false & \\
\hline
string   & ssid & & \\
\hline
datetime & observed\_at & & Informacja, kiedy dana sieć została zaobserwowana \\
\hline
float    & latitude & null: false & Szerokość geograficzna \\
\hline
float    & longitude & null: false & Wysokość geograficzna \\
\hline
datetime & geolocated\_at & & Gdy urządzenie rozróżnia czas ostatniej pozycji GPS od czasu obserwacji sieci bezprzewodowej \\
\hline
string   & source & default: "internal" & Nazwa źródła skąd przyszły dane. Jeżeli import to odpowiednia nazwa serwisu. Jeżeli z aplikacji mobilnej, to odpowiedni identyfikator urządzenia \\
\hline
string   & id\_of\_source & & Identyfikator w ramach pola source; Pozwoli na nie importowanie duplikatów lub możliwą aktualizację importowanych danych \\
\hline
json     & raw\_info & & Wszystkie atrybuty, jakie były zadeklarowane do obserwacji. Jeżeli byśmy importowali jakieś dane, które nie mają odpowiedniej kolumny w bazie danych, to w przyszłości możemy te dane wyciągnąć z tej kolumny. \\
\hline
datetime & created\_at & & Data i czas utworzenia rekordu (domyślnie dodawane przez Ruby on Rails) \\
\hline
datetime & updated\_at & & Data i czas modyfikacji rekordu (domyślnie dodawane przez Ruby on Rails) \\
\hline
\end{tabular}
\end{table}

Dodatkowym problemem stała się samo BSSID. W każdej z importowanych baz zostało zapisane w inny sposób. Okazało się, że w ramach importu bazy ze strony mylnikov.org sposób zapisu BSSID się różni. Czasem wszystkie litery były duże. Innym razem wszystkie były małe. Aby ujednolicić zapis, utworzyłem odpowiednią metodę klasową, którą potem wykorzystałem we wszystkich miejscach, skąd przyjmowałem informacje o BSSID.

\begin{minted}[breaklines=true]{ruby}
def self.standardize_bssid(value) # value = '00:0A:E6:3E:FD:E1'
  result = value.to_s.upcase.gsub(/[^A-F0-9]/, '')
  return nil if result.size != 12
  result
end
\end{minted}

Ze względu na prędkość importu zdecydowałem się zdenormalizować obserwacje punktów dostępów. Kolumna BSSID — opisującą adres sprzętowy urządzenia będzie posiadała powtarzające się dane. Dzięki temu mogłem użyć biblioteki activerecord-import. Która pozwala na import danych do bazy danych poprzez bardzo praktyczny interfejs, ale przede wszystkim w sposób bardzo szybki. Dzięki tej decyzji import informacji o lokalizacjach sieci bezprzewodowych dla obszaru Warszawy zajął prawie 6 godzin na moim komputerze.

\subsection{Prezentacja danych na mapie}
Do prezentacji danych, jakimi są lokalizacje sieci bezprzewodowych, wykorzystałem gotową bibliotekę Leaflet. Wykorzystywanie gotowych elementów interfejsu często przyspiesza pracę nad projektami. Zwłaszcza tak skomplikowanego interfejsu, jakim jest mapa. Sama biblioteka implementuje obsługę współrzędnych geograficznych, przybliżania i oddalania oraz prostych znaczków i kształtów geograficznych. Podstawowym elementem, jaki musiałem dodać to grafiki mapy — tzw. płytki map (z ang. map tiles). Mój wybór padł na dostawcę darmowych grafik — serwis CartoDB. Udostępniają oni wysokiej jakości, schludne grafiki do wyboru w kilku odcieniach. Na potrzeby tej pracy wybrałem minimalistyczną i ciemną wersję o nazwie dark. Do Leafleta dodałem jeszcze dwa dodatki: jeden odpowiedzialny za pobieranie danych standardem AJAX oraz drugi, który jest odpowiedzialny za klastrowanie dużej ilości znaczników na mapie w grupy.

Dla widoku mapy utworzyłem ścieżkę na serwerze, pod którą można pobrać znaczniki z lokalizacjami sieci bezprzewodowych. Zdecydowałem się na strukturę danych określoną w standardzie GeoJSON ze względu na kompatybilność z biblioteką Leaflet, dodatkowo stosując takie rozwiązanie, interfejs programistyczny jest bardziej przystępny do wykorzystania.

Ze względu na ogrom danych, jakimi są zapisane lokalizacje sieci bezprzewodowych, potrzebne jest wykorzystanie mechanizmu AJAX i ograniczenia przesyłanych danych do 25 tysięcy struktur GeoJSON. To ograniczenie wynika z wydajności przeglądarki do prezentacji tak dużej ilości danych. Więcej i przeglądarka Firefox miała kilkusekundowe przestoje podczas przeładowywania danych.

Samo generowanie struktury GeoJSON zaimplementowałem poprzez odpowiednie dwie metody:

\begin{minted}[breaklines=true]{ruby}
def geojson_hash
  {
    type: 'Feature',
    geometry: {
      type: 'Point',
      coordinates: geojson_coordinates
    },
    properties: {
      id: id,
      bssid: bssid
    }
  }
end

def geojson_coordinates
  [longitude, latitude]
end
\end{minted}

\subsection{Algorytm geolokalizacji}

Rozwiązanie tego problemu składa się z dwóch części. Pierwszą jest widok — perspektywa - w bazie danych, której zadaniem jest podanie lokalizacji dla najmocniej zarejestrowanego sygnału sieci bezprzewodowej. Dodatkowym elementem tego widoku jest sortowanie po roku oraz tygodniu, w którym sieć została zaobserwowana. Dzięki temu mamy pewność świeżości danych o lokalizacji sieci bezprzewodowej.

\begin{minted}[breaklines=true]{sql}
CREATE MATERIALIZED VIEW wifi_positions AS
  SELECT DISTINCT bssid, id, longitude, latitude,
      json_build_object(
        'type', 'Feature',
        'geometry', json_build_object(
          'type', 'Point',
          'coordinates', json_build_array(
            longitude,latitude
          )
        ),
        'properties', json_build_object(
          'id', id,
          'bssid', bssid,
          'ssid', ssid
        )
      )::text AS geojson, ssid
  FROM wifi_observations
  WHERE id = (
    SELECT wo.id
    FROM wifi_observations AS wo
    WHERE wo.bssid = wifi_observations.bssid
    ORDER BY
      (SELECT EXTRACT(YEAR FROM wo.observed_at)) DESC,
      (SELECT EXTRACT(WEEK FROM wo.observed_at)) DESC,
      COALESCE((raw_info->>'signal_level'),'100')::int ASC NULLS LAST
      LIMIT 1
\end{minted}

Widok ten jest odświeżany wyzwalaczem przy każdej operacji modyfikującej tabelę wifi\_observations.

\begin{minted}[breaklines=true]{sql}
create or replace function refresh_wifi_positions()
returns trigger language plpgsql
as \$\$
begin
    refresh materialized view wifi_positions;
    return null;
end \$\$;
create trigger refresh_wifi_positions_trigger
after insert or update or delete or truncate
on wifi_observations for each statement
execute procedure refresh_wifi_positions();
\end{minted}

Drugim elementem algorytmu jest obliczenie geograficznego środka ciężkości — punktu środkowego — dla znalezionych przez urządzenie lokalizacji sieci bezprzewodowych. Wykorzystałem do tego bibliotekę Geocoder.

\begin{minted}[breaklines=true]{ruby}
positions = WifiPosition.where(bssid: params[:wifiAccessPoints].map do |ap|
  WifiObservation.standardize_bssid(ap[:macAddress])
end)

Geocoder::Calculations.geographic_center(
  positions.map(&:coordinates))
\end{minted}

Warto jednak zajrzeć do implementacji metody geographic\_center, ponieważ znajdziemy tam algorytm, który przelicza współrzędne geograficzne na trójwymiarowy układ kartezjański, po czym wylicza średnią wartość wyznaczonych punktów, a na koniec przelicza z powrotem na współrzędne geograficzne.

\begin{minted}[breaklines=true]{ruby}
##
# Compute the geographic center (aka geographic midpoint, center of
# gravity) for an array of geocoded objects and/or [lat,lon] arrays
# (can be mixed). Any objects missing coordinates are ignored. Follows
# the procedure documented at http://www.geomidpoint.com/calculation.html.
#
def geographic_center(points)

  # convert objects to [lat,lon] arrays and convert degrees to radians
  coords = points.map{ |p| to_radians(extract_coordinates(p)) }

  # convert to Cartesian coordinates
  x = []; y = []; z = []
  coords.each do |p|
    x << Math.cos(p[0]) * Math.cos(p[1])
    y << Math.cos(p[0]) * Math.sin(p[1])
    z << Math.sin(p[0])
  end

  # compute average coordinate values
  xa, ya, za = [x,y,z].map do |c|
    c.inject(0){ |tot,i| tot += i } / c.size.to_f
  end

  # convert back to latitude/longitude
  lon = Math.atan2(ya, xa)
  hyp = Math.sqrt(xa**2 + ya**2)
  lat = Math.atan2(za, hyp)

  # return answer in degrees
  to_degrees [lat, lon]
end
\end{minted}

\subsection{Specyfikacja do Geolokalizacji użytkownika}
Interfejs do wykonania geolokalizacji przez aplikację internetową utworzyłem na podstawie specyfikacji produktu Google Maps.\cite{GoogleMapsGeolocationAPI} Oczywiście ta specyfikacja opisuje o wiele bardziej obszerne parametry do geolokalizacji. W ramach tej pracy zaimplementowałem parametry określające sieci bezprzewodowe: BSSID oraz siłę sygnału odebraną przez urządzenie.

\subsection{Widoki}
Wszystkie widoki aplikacji internetowej wykorzystują Boostrap oraz jQuery. Wykorzystanie tych bibliotek pozwala na szybkie tworzenie aplikacji, których użyteczność i estetyka są na zadowalającym poziome. Dodatkowo style Boostrapa są przystosowane do prezentacji widoków aplikacji w różnych rozdzielczościach ekranów. Nawet dla przeglądarek internetowych na smartfonach strona prezentuje się poprawnie i schludnie.
